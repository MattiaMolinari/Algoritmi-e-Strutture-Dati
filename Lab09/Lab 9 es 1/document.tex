\documentclass[11pt,a4paper]{report}
\usepackage[T1]{fontenc}
\usepackage[utf8]{inputenc}
\usepackage[italian]{babel}

\usepackage{hyperref}
\usepackage{authblk}
\usepackage[dvipsnames, table,xcdraw]{xcolor}
\usepackage[automark,
autooneside=false,% <- needed if you want to use \leftmark and \rightmark in a onesided document
headsepline
]{scrlayer-scrpage}
\usepackage[framemethod=TikZ]{mdframed}
\usepackage{epigraph} 
\usepackage{geometry}
\geometry{margin=3cm}
\usepackage{amsmath,amsfonts,amssymb,amsthm,epsfig,epstopdf,titling,url,array}
\usepackage{tikz}
\usepackage{mathrsfs}
\usepackage{listings}
\usepackage{multicol}
\setlength{\columnseprule}{1pt}
\def\columnseprulecolor{\color{black}}


\usepackage{caption}
\pagestyle{empty}


\lstset{extendedchars=false, language=c, showstringspaces=false, showspaces=false, showlines=false, showtabs = false, frame=single, basicstyle=\sffamily, moredelim=**[is][\color{red}]{§}{§},moredelim=**[is][\color{blue}]{@}{@},moredelim=**[is][\color{ForestGreen}]{£}{£},tabsize=4,morestring=*[b]", escapechar=`}



\begin{document}

\begin{center}
	\large	Mattia Molinari - 296544
	\end{center}
\text{\large \textbf{Laboratorio 9 esercizio 1: Grafi e DAG}}\\

La soluzione proposta è composta da cinque file:
\begin{itemize}
	\item \textbf{st.h}: definizione della tabella di simboli come ADT di I classe (tipo \lstinline|st_t|) e prototipi delle funzioni ad essa associate (vedere commenti per dettagli);

	\item \textbf{st.c}: implementazione della tabella di simboli secondo il modello del vettore non ordinato di stringhe e implementazione delle funzioni relative;
	\item \textbf{graph.h}: definizione del grafo come ADT di I classe (tipo \lstinline|graph_t|) e prototipi delle funzioni ad esso associate (vedere commenti per dettagli);
	\item \textbf{graph.c}: implementazione del grafo orientato come \lstinline|struct| con i seguenti campi:
	\begin{itemize}
		\item \lstinline|st_t st|: tabella di simboli, possibile grazie a \lstinline|#include "st.h"|
		\item \lstinline|edge_t *edges|: elenco degli archi, il tipo \lstinline|edge_t| viene dichiarato nel medesimo file come \lstinline|struct| che ha come campi l'indice del vertice di partenza, quello del vertice di arrivo e il peso dell'arco (tutti interi)
		\item \lstinline|int E, V|: numero di archi e vertici
		\item \lstinline|list_t *adj|: vettore di liste per le adiacenze, il tipo \lstinline|list_t| viene dichiarato nel medesimo file come \lstinline|struct| con puntatore a testa della lista
	\end{itemize}
	Risulta particolarmente rilevante l'implementazione delle seguenti funzioni:
	\begin{itemize}
		\item \lstinline|int GRAPHcheckDAG(graph_t G, int *mark, int *pre, int *post)|: funzione che viene opportunamente chiamata da main.c per controllare se la rimozione degli archi segnati in \lstinline|mark| sia un DAG o meno; per far ciò si utilizzano le funzioni \lstinline|GRAPHremoveEdges| e \lstinline|GRAPHaddEdges| rispettivamente per rimuovere e poi aggiungere nuovamente gi archi marcati alla lista delle adiacenze con un controllo intermedio di \lstinline|dfsR|: quest'ultima è stata opportunamente modificata affinché restituisca un valore se si trovano cicli [0] o meno [1] (vedere commenti per dettagli)
		\item \lstinline|void GRAPHmaximumPath(graph_t G, FILE *fout)|: funzione che utilizza \lstinline|TSdfsR| tratta dai lucidi per trovare l'ordine topologico dei vertici del DAG e calcolare le distanze massime di ognuno di essi da ogni sorgente
		
	\end{itemize}
	\item\textbf{ main.c}: file principale.
\end{itemize}
	
	La soluzione sfrutta il modello del calcolo combinatorio dell'\textbf{insieme delle parti} per generare insiemi di archi: in particolare si utilizza il modello che richiama le \underline{combinazioni semplici} in quanto permette la generazione di insiemi di cardinalità crescente e dunque di trovare prima quelli di cardinalità minima.
	Più precisamente i passaggi sono i seguenti:
	\begin{enumerate}
		\item inizializzazione variabili utili e grafo \lstinline|G| caricando le informazioni da file;
		\item allocazione variabili: da notare il vettore \lstinline|mark|, la cui posizione $i$-esima marca l'arco di indice $i$ nel vettore \lstinline|edges| (visibile solo in graph.c) come appartenente [1] o meno [0] all'insieme;
		\item si controlla che il grafo non sia già un DAG (se lo è si stampa un messaggio a terminale e si chiede se si vuole proseguire comunque);
		\item si utilizza la funzione \lstinline|powerset| in un opportuno ciclo per generare gli insiemi partendo da quelli a cardinalità minore;
		\item all'interno della funzione \lstinline|powerset| si controlla per ogni insieme generato se la rimozione degli archi marcati genererebbe un DAG: in caso di risposta affermativa, si aggiorna il flag ad 1 e si stampa la cardinalità \lstinline|k| raggiunta, per poi stampare l'insieme corrente; la funzione prosegue generando solo più gli insiemi con cardinalità \lstinline|k|, stampando e confrontando il peso totale con \lstinline|*maxWt| solo di quelli che se rimossi generano un DAG (si aggiorna eventualmente la soluzione ottima \lstinline|sol|);
		\item terminata la ricorsione e il ciclo \lstinline|for|, si stampa la soluzione finale con relativo peso;
		\item si rimuovono definitivamente gli archi marcati e si stampano le distanze massime da ogni nodo sorgente verso ogni nodo del DAG;
		\item liberazione spazio allocato.
	\end{enumerate}
	


\end{document}